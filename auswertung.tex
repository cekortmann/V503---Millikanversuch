\section{Auswertung}
\label{sec:Auswertung}

\subsection{Fehlerrechnung}
\label{sec:Fehlerrechnung}
Für die Fehlerrechnung werden folgende Formeln aus der Vorlesung verwendet.
für den Mittelwert gilt
\begin{equation}
    \overline{x}=\frac{1}{N}\sum_{i=1}^N x_i ß\; \;\text{mit der Anzahl N und den Messwerten x} 
    \label{eqn:Mittelwert}
\end{equation}
Der Fehler für den Mittelwert lässt sich gemäß
\begin{equation}
    \increment \overline{x}=\frac{1}{\sqrt{N}}\sqrt{\frac{1}{N-1}\sum_{i=1}^N(x_i-\overline{x})^2}
    \label{eqn:FehlerMittelwert}
\end{equation}
berechnen.
Wenn im weiteren Verlauf der Berechnung mit der fehlerhaften Größe gerechnet wird, kann der Fehler der folgenden Größe
mittels Gaußscher Fehlerfortpflanzung berechnet werden. Die Formel hierfür ist
\begin{equation}
    \increment f= \sqrt{\sum_{i=1}^N\left(\frac{\partial f}{\partial x_i}\right)^2\cdot(\increment x_i)^2}.
    \label{eqn:GaussMittelwert}
\end{equation}


\subsection{Tröpfchengeschwindigkeiten}
Für zwei verschiedene Spannungen wurde sowohl im eingeschalteten elektrischen Feld die Auf- und Abstiegszeit des jeweiligen Tröpfchens und bei ausgeschaltetem 
Feld die Zeit der Gleichgewichtsgeschwindigkeit gemessen. Für eine Spannung von $U = 201\,\unit{\volt}$ sind die Messdaten der einzelnen 
Tröpfchen in \autoref{tab:200} dargestellt, für eine Spannung von $U = 250\,\unit{\volt}$ in \autoref{tab:250}. Die Messwerte, die mit einem 
Sternchen markiert wurden, wurden in den handschriftlichen Messdaten falsch aufgeschrieben (siehe Anhang).

\begin{table}
    \centering
    \caption{Messdaten der Auf- und Abstiegszeit bei $U=201\,\unit{\V}$.}
\begin{tabular}{c c c c}
    \toprule
        Tröpfchen &$t_{\symup{auf}} \mathbin{/} \unit{\s}$ & $t_{\symup{ab}}\mathbin{/} \unit{\s}$ & $t_0 \mathbin{/}\unit{\s}$ \\
    \midrule
    1&7.19&3.29&12.57\\
    &8.02&3.65&\\
    &7.91&4.51&\\
    2&3.00&2.97&37.04\\
    &3.19&3.05&\\
    &3.28&1.02&\\
    3&1.43&1.24&73.73\\
    &1.60&1.10&\\
    &1.18&1.38&\\
    4&1.28&1.00&72.05\\
    &1.11&1.28&\\
    &1.47&1.30&\\
    5&5.38&2.06&26.80\\
    &1.95&2.32&\\
    &3.50&3.17&\\
    6&1.82&2.80&62.36\\
    &7.80&2.18&\\
    &8.50&2.94&\\
    7&3.72&9.29&31.75\\
    &4.11&8.29&\\
    &3.24*&14.39*&\\
    8&6.77&8.21&20.41 \\
    &10.65&8.51&\\
    &12.86&5.81&\\
    9&2.04*&1.50*&72.52\\
    &2.13&1.78&\\
    &2.33&1.97&\\
    10&2.06&1.62&73.85 \\
    &1.99&1.79&\\
    &2.45&1.78& \\
    \bottomrule
    \end{tabular}
    \label{tab:200}
\end{table}

\begin{table}
    \centering
    \caption{Messdaten der Auf- und Abstiegszeit bei $U=250\,\unit{\V}$.}
\begin{tabular}{c c c c}
    \toprule
        Tröpfchen &$t_{\symup{auf}} \mathbin{/} \unit{\s}$ & $t_{\symup{ab}}\mathbin{/} \unit{\s}$ & $t_0 \mathbin{/}\unit{\s}$ \\
    \midrule
    1&1.88&1.51&92.45 \\
    &5.67&1.82& \\
    &2.35&5.06& \\
2&4.99&5.17&105.44 \\
    &5.06&4.62& \\
    &3.19&4.57& \\
  3&4.98&4.93&/ \\
     &5.79&4.40& \\
    &4.69&4.08& \\
4&8.85&4.55&21.89 \\
    &8.19&4.94& \\
     &8.50&4.93& \\
  5&4.88&5.47&/ \\
     &5.25&5.90& \\
    &5.03&5.43& \\
6&11.88&7.72&48.65 \\
   &10.59&8.43& \\
    &9.89&7.82& \\
    \bottomrule
    \end{tabular}
    \label{tab:250}
\end{table}

\newpage
Nun werden die Abstiegs- und Aufstiegsgeschwindigkeiten sowie die Gleichgewichtsgeschwindigkeit $v_0$ mittels
\begin{equation*}
    v=\frac{s}{t}
\end{equation*} 
berechnet. Dabei ist $s=0.5\,\unit{\milli\m}$. Des Weiteren werden die Geschwindigkeiten $v_{\symup{auf}}$ und $v_{\symup{ab}}$ für jeden Tropfen gemittelt und dann
dessen Differenz $\bar{v}_{\symup{ab}}-\bar{v}_{\symup{auf}}$ betrachtet. Alle Werte werden in \autoref{tab:200geschw} dargestellt. 
Die Differenz $\bar{v}_{\symup{ab}}-\bar{v}_{\symup{auf}}$ wird durch die Relation
\begin{equation}
    \bar{v}_{\symup{ab}}-\bar{v}_{\symup{auf}} = 2v_0
    \label{eqn:gültigkeit}
\end{equation}
mit der Gleichgewichtsgeschwindigkeit verglichen. Dabei fällt auf, dass nur der erste und der fünfte gemessene Tropfen die Bedingung erfüllen. 



\begin{sidewaystable}
    \centering
    \caption{Messdaten der Auf- und Abstiegsgeschwindigkeit bei $U=201\,\unit{\V}$.}
\begin{tabular}{c c c c c c c c}
    \toprule
        Tröpfchen &$v_{\symup{auf}} \mathbin{/} 10^{-3}\unit{\m\per\s}$ & $v_{\symup{ab}}\mathbin{/} 10^{-3}\unit{\m\per\s}$ & $\bar{v}_{\symup{auf}} \mathbin{/} 10^{-3}\unit{\m\per\s}$& $\bar{v}_{\symup{ab}} \mathbin{/} 10^{-3}\unit{\m\per\s}$ & $\bar{v}_{\symup{ab}} -\bar{v}_{\symup{ab}} \mathbin{/} 10^{-3}\unit{\m\per\s}$ & $v_0 \mathbin{/}10^{-3}\unit{\m\per\s}$ & $2v_0 \mathbin{/}10^{-3}\unit{\m\per\s}$\\
    \midrule
    1&0.0695&0.1520&0.0650&0.1333&0.0682&0.0398&0.0796\\
    &0.0623&0.1370&&&&& \\
    &0.0632&0.1109&&&&& \\
2&0.1667&0.1684&0.1586&0.2742&0.1155&0.0135&0.0270\\
    &0.1567&0.1639&&&&& \\
    &0.1524&0.4902&&&&& \\
3&0.3497&0.4032&0.3620&0.4067&0.0447&0.0068&0.0136\\
    &0.3125&0.4545&&&&& \\
    &0.4237&0.3623&&&&& \\
4&0.3906&0.5000&0.3937&0.4251&0.0313&0.0069&0.0139\\
    &0.4505&0.3906&&&&& \\
    &0.3401&0.3846&&&&& \\
5&0.0929&0.2427&0.1641&0.2053&0.0413&0.0187&0.0373\\
    &0.2564&0.2155&&&&& \\
    &0.1429&0.1577&&&&& \\
6&0.2747&0.1786&0.1326&0.1927&0.0601&0.0080&0.0160\\
    &0.0641&0.2294&&&&& \\
    &0.0588&0.1701&&&&& \\
7&0.0538&0.1344&0.0496&0.1368&0.0872&0.0157&0.0315 \\
    &0.0603&0.1217&&&&& \\
   &0.0347&0.1543&&&&& \\
8&0.0739&0.0609&0.0532&0.0686&0.0153&0.0245&0.0490 \\
    &0.0469&0.0588&&&&&\\
    &0.0389&0.0861&&&&&\\
9&0.2451&0.3333&0.2315&0.2893&0.0579&0.0069&0.0138\\
    &0.2347&0.2809&&&&&\\
    &0.2146&0.2538&&&&&\\
10&0.2427&0.3086&0.2327&0.2896&0.0569&0.0068&0.0135\\
    &0.2513&0.2793&&&&&\\
    &0.2041&0.2809&&&&&\\
    \bottomrule
    \end{tabular}
    \label{tab:200geschw}
\end{sidewaystable}

Von diesen beiden wird nun mittels \autoref{eqn:r1} der Tröpfchenradius berechnet. Dabei wird die temperaturabhängige Viskosität von Luft benötigt, dessen Werte 
in \autoref{tab:Viskositaet} aufgetragen sind. Für die Dichte von Luft wird $\rho_{\symup{L}} = 1.024 \,\unit{\kg\per\m^3}$ \cite{luftdichte} 
und für die Dichte von Öl wird $\rho_{\symup{Öl}} = 886 \,\unit{\kg\per\m^3}$ \cite{ap503} angenommen.



\begin{sidewaystable}
    \centering
    \caption{Messdaten der Auf- und Abstiegsgeschwindigkeit bei $U=250\,\unit{\V}$.}
\begin{tabular}{c c c c c c c c}
    \toprule
        Tröpfchen &$v_{\symup{auf}} \mathbin{/} 10^{-3}\unit{\m\per\s}$ & $v_{\symup{ab}}\mathbin{/} 10^{-3}\unit{\m\per\s}$ & $\bar{v}_{\symup{auf}} \mathbin{/} 10^{-3}\unit{\m\per\s}$& $\bar{v}_{\symup{ab}} \mathbin{/} 10^{-3}\unit{\m\per\s}$ & $\bar{v}_{\symup{ab}} -\bar{v}_{\symup{ab}} \mathbin{/} 10^{-3}\unit{\m\per\s}$ & $v_0 \mathbin{/}10^{-3}\unit{\m\per\s}$ & $2v_0 \mathbin{/}10^{-3}\unit{\m\per\s}$\\
    \midrule
    11&0.2660&0.3311&0.1900&0.2350&0.0459&0.0054&0.0108 \\
              &0.0882&0.2747&&&&& \\
              &0.2128&0.0988&&&&& \\
    12&0.1002&0.0967&0.1028&0.1206&0.0178&0.0047&0.0095 \\
              &0.0988&0.1082&&&&& \\
              &0.1094&0.1567&&&&& \\
    13&0.1004&0.1014&0.0978&0.1125&0.0147& NaN & NaN\\
              &0.0863&0.1136&&&&& \\
               &0.1066&0.1225&&&&& \\
    14&0.0565&0.1099&0.0587&0.1042&0.0454&0.0228&0.0457 \\
              &0.0611&0.1012&&&&& \\
              &0.0588&0.1014&&&&& \\
    15&0.0914&0.1025&0.0894&0.0990&0.0096& NaN& NaN \\
              &0.0847&0.0952&&&&& \\
              &0.0921&0.0994&&&&& \\
    16&0.0421&0.0648&0.0466&0.0627&0.0161&0.0103&0.0206 \\
              &0.0472&0.0593&&&&& \\
            &0.0506&0.0639&&&&& \\
    \bottomrule
    \end{tabular}
    \label{tab:2500geschw}
\end{sidewaystable}
\begin{table}
    \centering
    \caption{Lufttemperaturen und Luftviskositäten \cite{ap503}.}
\begin{tabular}{c c c}
    \toprule
        $R\mathbin{/}\unit{\mega\ohm}$ &$T\mathbin{/}\unit{\celsius}$ & $\eta_{\symup{L}}\mathbin{/}10^{-5}\symup{Nsm^{-2}}$ \\
    \midrule
    2.05 & 24.0 & 1.844 \\
    2.08 & 23.5 & 1.840\\
    2.17 & 22.0 & 1.834\\
    2.20 & 21.5 & 1.832\\
     \bottomrule
    \end{tabular}
    \label{tab:Viskositaet}
\end{table}

Damit berechnen sich die Radien zu
\begin{align*}
    r_1 &= 1.6723 \cdot 10^{-7} \,\unit{\m} \; ,\\
    r_5 &= 2.9967 \cdot 10^{-7} \,\unit{\m}\; .
\end{align*}
Für die zweite Messreihe mit einer Kondensatorspannung von $250\, \unit{\volt}$ war es teilweise nicht möglich eine Gleichgewichtsgeschwindigkeit $v_0$ zu bestimmen, da die Tröpfchen sich ohne E-Feld nicht bewegt haben.
Aus der zweiten Messreihe sind somit nur 2 Werte, die die Relation aus \autoref{eqn:gültigkeit} erfüllen.
Die weiteren Berechnungen sind somit nur mit Daten dieser Tropfen valide.
%Nach \autoref{eqn:r1} ergeben sich die Radien der beiden Tröpfchen zu 
%\begin{align*}
%    r_{14}& = 1.4764\cdot 10^{-5}\, \unit{\meter}\\
%    r_{16}& = 9.9231\cdot 10^{-6}\, \unit{\meter}\. .
%\end{align*}
Das Bestimmen des Radius ist über \autoref{eqn:r2} möglich.
Somit folgen für die Radien 
\begin{align*}
    r_{14}& = 1.4748\cdot 10^{-7}\, \unit{\meter}\\
    r_{16}& =  5.1558\cdot 10^{-7}\, \unit{\meter}\. .
\end{align*}

Dabei wurde ein Luftdruck von $1.013\,\unit{\milli\bar}$ verwendet \cite{ap503}.

Die Ladungen der jeweiligen Tröpfchen berechnen sich zu mittels \autoref{eqn:q} und werden in \autoref{tab:q} dargestellt. Es wird ebenfalls die
korrigierte Ladung mittels \autoref{eqn:qeff} berechnet.
\begin{table}
    \centering
    \caption{Ladungen und korrigierte Ladungen.}
\begin{tabular}{c c c }
    \toprule
        Tröpfchen & $q \mathbin{/} 10^{-19}\unit{\coulomb}$ &$q_{\symup{korr}} \mathbin{/} 10^{-14}\unit{\coulomb}$  \\
    \midrule
    1& 12.183 \pm 0.008 & 22.059 \pm 0.015 \\
    5 & 7.117 \pm 0.005 & 10.198 \pm 0.007 \\
    14 & 4.0269 \pm 0.003 & 7.776 \pm 0.005 \\
    16 & 56.11 \pm 0.04 & 69.88 \pm 0.05 \\
    \bottomrule
    \end{tabular}
    \label{tab:q}
\end{table}

In \autoref{fig:ladungkorr} werden nun zur Bestimmung der Elementarladung die korrigierten Ladungen aufgetragen.
\begin{figure}
    \centering
    \includegraphics[height = 12cm]{build/Ladung.pdf}
    \caption{Korrigierte Ladungen.}
    \label{fig:ladungkorr}
\end{figure}
Die Ladungen wurden ihrer Größe nach sortiert und in so zugeteilt, dass eine lineare Ausgleichskurve durch diese gelegt werden kann.
Nach zunächst groben Sortieren folgte die genauere Einordnung der Ladungen in ein Vielfaches von dem größten gemeinsamen Teiler, welcher dann dem Wert der gesuchten Elementarladung $e_0$ entspricht.
Die Steigung dieser Ausgleichsgerade entspricht demnach der gesuchten Größe.
Der graphisch bestimmte Wert für die Elementarladung liegt bei $e_0=2.417\cdot 10^{-19}\,\unit{\coulomb}$ und entspricht somit nicht ganz dem theoretischen Wert ($e_{0,\text{theo}}=1.602\cdot 10^{-19}\,\unit{\coulomb}$), 
welcher ebenfalls in \autoref{fig:ladungkorr} dargestellt ist.
Die Avogadro-Konstant wird über die Formel 
\begin{equation*}
    N_{\text{A}}=\frac{F}{e_0} 
\end{equation*}
wobei F die Faraday-Konstante mit einem Wert $9.648\cdot 10^4\, \unit{\frac{\coulomb}{\mol}}$ ist.
Aus der oben bestimmten Ladung folgt so die Avogadro-Konstante
\begin{equation*}
    N_{\text{A,exp}}=\frac{F}{2.417\cdot 10^{-19}\,\unit{\coulomb}}= 3.992\cdot 10^{23}\, \unit{\mol}^{-1}
\end{equation*}