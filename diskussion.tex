\section{Diskussion}
\label{sec:Diskussion}

Wenn der unkorrigierte Wert der Elementarladung mit dem Literaturwert verglichen wird, lässt sich eine Abweichung von $3.00 \,\%$ erkennen. Die
experimentell bestimmte Elementarladung stimmt im Rahmen der angegebenen Messunsicherheit mit dem theoretischen Wert überein. Jedoch befinden sich nach der
Korrektur zwei Ladungen unterhalb der Elementarladung, was so eigentlich nicht vorkommen dürfte. Dies kann aus Messfehlern resultieren. Gerade die Reaktionszeit beim Messen der Zeit hat einen großen Einfluss auf die Messdaten, gerade weil
das zu messende Zeitintervall teilweise sehr gering war. Desweiteren war es bei vielen Tropfen im Bild teilweise nur schwierig möglich, den gewählten 
Tropfen über mehrere Zeitintervalle nicht aus den Augen zu verlieren, sodass nicht auszuschließen ist, dass bei einer Messung verschiedene Tropfen 
gemessen wurden. Insgesamt gab es nur fünf Tropfen, die die nötige Bedingung erfüllten, sodass das Ergebnis statistisch noch nicht wirklich aussagekräftig ist.

Die experimentell bestimmte Avogadrokonstante weicht um $2.86\,\%$ von dem Literaturwert $N_{\symup{A}}=6.0221 \cdot10^{23} \,\symup{\frac{1}{mol}}$ 
\cite{avo} ab, stimmt aber ebenfalls im Rahmen der angegebenen Messunsicherheit überein.