\section{Durchführung}
\label{sec:Durchführung}

Vor Beginn des Versuchs muss mit Hilfe der Libelle überprüft werden, ob die Messapparatur 
waagerecht steht. Außerdem wird kontrolliert, ob die Spannungsquelle am Plattenkondensator 
anliegt.

Zuerst müssen bei abgeschaltetem Plattenkondensator Öltröpfchen in die Kammer gesprüht 
werden. Dabei ist wichtig, dass nicht zu viel Öl in die Kammer gelangt. Nun wird 
ein relativ langsames Öltröpfchen ausgewählt, welches im Anschluss über eine selbst 
gewählte Strecke beobachtet wird. Vor dessen Untersuchung, muss überprüft werden, ob das 
Tröpfchen geladen ist. Dies geschieht, indem man das elektrische Feld eingeschaltet 
wird und einmal umgelegt wird. Wenn es bei dem umgepolten Feld keine Veränderung gegenüber 
vorher gibt, ist das Teilchen ungeladen. In diesem Fall muss bei ausgeschaltetem 
Feld, das radioaktive Präparat aktiviert werden, um das Tröpfchen entsprechend zu 
laden. 

Jetzt wird mit der eigentlichen Messung begonnen. Dazu wird die Fallgeschwindigkeit $v_{\symup{ab}}$
des Tröpfchens gemessen, indem die Zeit gemessen wird, in der das Tröpfchen 
eine bestimmte Strecke zurücklegt. Dann wird das Feld umgepolt und die 
Steiggeschwindigkeit $v_{\symup{auf}}$ gemessen. Diese Messung wird bei dem selben 
Tröpfchen so oft wie möglich wiederholt. Zuletzt wird die Gleichgewichtsgeschwindigkeit 
$v_0$ bei ausgeschaltetem elektrischen Feld gemessen. 

Die gesamte Messung wird für mehrere Öltröpfchen wiederholt. Außerdem werden für insgesamt 
zwei verschiedene Kondensatorspannungen jeweils mehrere Öltröpfchen beobachtet.
Dabei darf die Spannung an den Kondensatorplatten $500\,\unit{\volt}$ nicht übersteigen.